\documentclass[runningheads]{llncs}

%\usepackage{times}
%\usepackage{epsfig}
\usepackage{graphicx}
\usepackage{caption}
%\usepackage{subfigure}
\usepackage{subcaption}
\captionsetup{compatibility=false}
\usepackage{amsmath}
\usepackage{amssymb}
\usepackage{ruler}
\usepackage{color}
%\usepackage{pgfplots}
\usepackage[width=122mm,left=12mm,paperwidth=146mm,height=193mm,top=12mm,paperheight=217mm]{geometry}
%\newtheorem{proposition}{Proposition}
%\newtheorem{lemma}{Lemma}
%\newtheorem{proof}{Proof}
\newcommand{\RR}{\mathbb R}
\newcommand{\NNN}{\mathcal N}
\newcommand{\sumi}{\displaystyle{\sum_{i=1}^n}}
\DeclareMathOperator*{\argmin}{argmin}

% Highlighting
\usepackage{soul}
\usepackage{color}
%\usepackage[usenames,dvipsnames]{xcolor}
\newcommand{\hlc}[2][yellow]{{\sethlcolor{#1}\hl{#2}}}
\newcommand{\RAF}[1]{\hlc[yellow]{(RR:) #1}}
\newcommand{\JZ}[1]{\hlc[pink]{(JZ:) #1}}
\newcommand{\PP}[1]{\hlc[green]{PP: #1}}

\usepackage{tikz}
\usepackage{pgfplots,pgfplotstable}
\usetikzlibrary{pgfplots.groupplots}
\pgfplotsset{grid=major,height=2in, width=\columnwidth}
\input{tikz_styles.tex}

%\input{test_tikz_data.tex}

\begin{document}
% \renewcommand\thelinenumber{\color[rgb]{0.2,0.5,0.8}\normalfont\sffamily\scriptsize\arabic{linenumber}\color[rgb]{0,0,0}}
% \renewcommand\makeLineNumber {\hss\thelinenumber\ \hspace{6mm} \rlap{\hskip\textwidth\ \hspace{6.5mm}\thelinenumber}}
% \linenumbers


\pagestyle{headings}
\mainmatter
\def\ECCV16SubNumber{1390}  % Insert your submission number here

\title{Kernel Squared Loss Exemplar Machines For Image Retrieval} % Replace with your title

\titlerunning{ECCV-16 submission ID \ECCV16SubNumber}

\authorrunning{ECCV-16 submission ID \ECCV16SubNumber}

\author{Anonymous ECCV submission}
\institute{Paper ID \ECCV16SubNumber}


\maketitle
%\thispagestyle{empty}

%%%%%%%%% ABSTRACT
\begin{abstract}
This paper proposes an extension to the exemplar SVM (ESVM) feature encoding pipeline first proposed by Zepeda and P\'erez \cite{ZePe15}. We first show that, by replacing the hinge loss by the square loss in the ESVM cost function, similar results in image retrieval can be obtained on a fraction of the computational cost. We call this model square loss exemplar machine, or SLEM. Secondly, we introduce a kernelized SLEM variant which benefits from the same computational advantages but displays improved performance. Both SLEM variants exploit the fact that the negative examples are biased for all the positives in the training set, so most of the SLEM computational complexity can be incurred offline by exploiting the Woodbury matrix identity. Our experiments establish the performance and computational advantages of our methods using a large array of state-of-the-art base feature representations, and well-known image retrieval dataset.
\end{abstract}

\section{Introduction}

Robust image representations are a crucial component of a vast number of computer vision applications. Within these, the \emph{image retrieval} application is an important example wherein a query image is provided to the system, which must in turn find all matching images from within a large, unannotated database. The matching process is done entirely based on the pixel content of the query and database images, and the search must be robust to large image variations in camera pose, color and scene illumination, amongst others.

The success of several existing systems in this challenging application has indeed been enabled by advances in  image representation functions. An image representation function commonly maps a given input image to a vector in a fixed dimensional space called the image \emph{feature vector}. The image retrieval process then reduces to finding the feature vectors closest to the query image feature vector under {\it e.g.,} the Euclidean distance. An adequate image representation function must hence produce feature vectors that are robust to image variations, while at the same time incurring low computational overhead.

A now pervasive example of an image feature representation is that consisting of the activation coefficients extracted from the previous-to-last layer of Convolutional Neural Networks (CNNs) \cite{}. Although CNNs are trained in a fully-supervised manner for the image classification task, they have been shown to transfer well not only to new, unseen classes \cite{Oquab,Kulkarni,ChatfieldDevil}, but also to alternate tasks such as object detection \cite{RCNN} and, importantly, image retrieval \cite{AstoundingBaseline}.

Yet the more successful current feature representations for image retrieval rely on generative models such as $K$-means \cite{VLAD} or Gaussian Mixture Models \cite{Fisher}. Very few methods \cite{NetVLAD,Aakanksha,Cagdas} exist that exploit supervised learning of image features directly for the image retrieval task. One of the main reasons for this is the lack of adequately large and varied supervised datasets that are expensive to collect \cite{NetVLAD}.

The Exemplar Support Vector Machine (E-SVM) formulation originally proposed by Malisiewicz {\it et al.} \cite{Malisiewicz} is a way to leverage the availability of large, unannotated pools of images within the context of supervised learning. The approach consists of using a large generic pool as a set of negative examples, while using a single image (the \emph{exemplar}) as a positive example. Given these training set, a linear classifier is learned that can  generalize well, despite the drastically limited size of the set of positive examples. 

An extension \cite{ZePe15} of the E-SVM formulation discussed above instead treats the resulting linear classifier as an \emph{enhanced} representation of the image feature representing the exemplar image. Enhanced representations can thus be extracted from each database image, as well as from the query image, by treating each image as an exemplar while keeping a fixed pool of generic negative images. Searching hence amounts to computing distances between these new, enhanced E-SVM features. An interesting aspect of this approach is that the enhanced representation extracts what is unique about the exemplar relative to the pool of generic negatives.



Image search
- Importance.
- Feature representations. 

Feature representation - supervised learning
- Generic framework.
- Deep CNN features.
- ImageNet.

Image retrieval
- Supervised - Eusipco, NetVLAD.
- VLAD, Fisher.
- Cheaper.

Pool:
LDA
SVMs for feature representations - parts learning, Malizsiewics


%\section {Introduction}
% The exemplar SVM (E-SVM) was first introduced by Malisiewicz et al. in \cite{Efros11} as a conceptually simple framework for object detection and image classification where the training set has a small ratio of positive/negative examples. 
% At training time, many SVMs are learned from a large pool of negative against a single positive (so called exemplar). 
% At test time, the scores of the test images for each classifier are fitted by a logistic regression. 
% The final score of an image is then a non-linear combination of scores from multiples exemplar SVMs.

% They have also been used in \cite{Efros12} in image retrieval tasks, using the classifier score to rank matching candidates.
% However this transfer of information from classification to retrieval is severely limited. 
% Indeed, the purpose of a support vector machine is to separate positive and negative samples, i.e., predict discrete labels. 
% The distance between a negative sample and the classification hyperplane has no value as a measure of has \emph{a priori} no value of continuous matching score.

% Zepeda et al. address this problem in \cite{ZePe15} by firstly, performing an E-SVM to each image in a dataset instead of only for the query images and secondly, comparing its classifiers distance to the query's classifier instead of comparing scores. 
% These modifications guarantee we are ranking distances between two points instead of classification scores.
% Therefore \cite{ZePe15} refers to E-SVMs as features encoders: a pipeline that takes an image representation as input and returns an improved image representation. This type of feature enhancing is more akin to methods such as whitening, PCA and LDA.

% This paper introduces the square-loss Exemplar machine (SLEM), which consists of optimizing the same cost function of a regular E-SVM where the hinge loss is replaced by the square loss. 
% The square loss version has the advantage of having a much more efficient optimization. Indeed, the minimization of its cost function is solved by a linear system and can be done for all exemplar simultaneously, whereas the regular E-SVM cost function is generally minimized one exemplar at a time, normally by stochastic gradient descent \cite{bottou10}.
% Also, for the machine learning tasks of binary classification, both regular SVM and least-squares SVM have similar performances \cite{YeXi07}.
% %Also, for the machine learning tasks of binary classification, both regular SVM and least-squares SVM have the same performance if positive and negative samples are separable and the pool of samples is linearly independent \cite{YeXi07}. 
% We also introduce a kernelized version of SLEM, efficiently implemented, that gives better results and superior scalability for large-scale image retrieval problems.

%%% Local Variables:
%%% TeX-master: "main_eccv"
%%% End:

%\input{sec_prior.tex}

\section{Square loss exemplar machine}\label{lsesvm}
In this section, we revisit the exemplar SVM model as presented by \cite{Efros11} and introduce its square loss version.
\subsection{Exemplar SVMs} \label{esvm}
Given features in $\RR^d$ at training time, one positive example $x_0$ in $\RR^d$ and a set of negative examples $X = [x_1, x_2,...,x_n]$ in $\RR^{d\times n}$, each column of $X$ representing one example by a vector in $\RR^d$. 
Any feature vector representation can be used.
We are also given a loss function $l:\{-1, 1\}\times \RR\rightarrow\RR^+$. Learning an E-SVM from these examples amounts to minimizing the function 
\begin{equation}
J(\omega, \nu) = \theta \ l(1, \omega^Tx_0+\nu) +\dfrac{1}{n}\sumi l(-1, \omega^T x_i+\nu)+\dfrac{\lambda}{2}|\!|\omega|\!|^2, \label{eq:first}
\end{equation}
w.r.t. $\omega$ in $\RR^d$ and $\nu$ in $\RR$.

In Equation (\ref{eq:first}), $\lambda$ and $\theta$ are respectively a regularization parameter on $\omega$ and a positive scalar adjusting the weight of the positive exemplar.
%\footnote{We could also acknowledge a parameter $\theta$ other than $\frac{1}{n}$ as a regularization to the error of each negative example. But this parameters seems to be less important to cross-validate. Also, setting $\theta=\frac{1}{n}$ simplify Equation (\ref{omega:solution}) and allows the use of Woodbury identity.} 

The  exemplar SVM of $x_0$ with respect to $X$ is the classifier $\omega^\star(x_0,X)$ that minimizes the loss function $J$:
\begin{equation}
\big(\omega^\star(x_0, X), \nu^\star(x_0, X)\big) = \underset{(\omega,\nu)\in\RR^d\times\RR}{\argmin} \ J(\omega, \nu). \label{omega:first}
\end{equation}
To shorten the notation, we refer to these solutions as $\omega^\star$ and $\nu^\star$ when their arguments are implicit.

Applications of E-SVM use the hinge loss function, which guarantees that $J$ is a convex function. The solution of Equation (\ref{omega:first}) can be thus found by stochastic gradient descent.

\subsection{Square loss function}\label{SLEM}
Now, let us study the same learning problem for the square-loss function $l(y,\hat{y}) = \frac{1}{2}(y-\hat{y})^2$. As for the hinge loss, Equation (\ref{eq:first}) is a convex problem. 
However, differently from the hinge loss, it is now a ridge regression problem, whose solution is expressed in a closed form:
\begin{align}
\begin{cases}
\vspace{3 mm}
\omega^\star &= \dfrac{2\theta}{\theta+1}U^{-1}(x_0-\mu), \\
%\vspace{3 mm}
\nu^\star &= \dfrac{\theta-1}{\theta+1}-\dfrac{1}{\theta+1}(\theta x_0+\mu)^T\omega^\star,
\end{cases}
\label{omega:solution}
\end{align}
%\begin{align}
%\nu^\star &= \dfrac{\theta-1}{\theta+1}-\dfrac{1}{\theta+1}(\theta x_0+\mu)^T\omega^\star, \label{eq:nustar}\\
%\omega^\star &= \dfrac{2\theta}{\theta+1}U^{-1}(x_0-\mu) \label{eq:wstar}
%\end{align}
where:
\begin{align}
&\mu = \frac{1}{n}\sum_{i=1}^n x_i,\\
&U = \dfrac{1}{n}XX^T-\mu\mu^T+\dfrac{\theta}{\theta+1}(x_0-\mu)(x_0-\mu)^T+\lambda\textbf{Id}_d. \label{eq:U}
\end{align}
Equation \ref{omega:solution} shows how to solve (\ref{eq:first}) in a closed form to obtain the SLEM vector $\omega^\star(x_0,X)$ of $x_0$. 
Replacing the hinge loss by the square loss offers a more compact solution, but not necessarily more efficient.

\subsection{Woodbury identity and matching scores}
Let us define $A = \frac{1}{n}XX^T-\mu\mu^T +\lambda\textbf{Id}_d$ and assume $A^{-1}$ known. 
%$A$ is the covariance matrix of the negative samples $X$\footnote{added of a small term proportional to the identity matrix to insure $A$ is positive-definite.}. Let us assume its inverse $A^{-1}$ is known. 
Matrix $U$ now reads $U = A + \frac{\theta}{\theta+1}\delta\delta^T$, where $\delta=x_0-\mu$. The Woodbury identity \cite{woodbury} gives us
\begin{equation}
U^{-1} = A^{-1} -\dfrac{\theta}{\theta\delta^TA^{-1}\delta+ \theta+1}A^{-1}\delta^T\delta A^{-1}. \label{invU}
\end{equation}

Using (\ref{invU}) in (\ref{omega:solution}) yields
\begin{equation}
\begin{split}
\omega^\star &= \dfrac{2\theta}{\theta+1}U^{-1}\delta \\
&= A^{-1}\delta - \dfrac{\theta}{\theta\delta^TA^{-1}\delta+ \theta+1} A^{-1}\delta (\delta^TA^{-1}\delta)\\
&= \dfrac{2\theta}{\theta\delta^TA^{-1}\delta+ \theta+1} A^{-1}\delta.\label{Wood:omega}
\end{split}
\end{equation}

The first observation we make from Equation (\ref{Wood:omega}) is that the positive sample weight $\theta$ does not influence the direction of the optimal vector $\omega^\star$, only its norm. This means that if search and ranking are based on the cosine-similarity, as in \cite{ZePe15}, $\theta$ does not influence the matching score of the SLEM vectors of two different images.
Indeed, if $\omega$ and $\omega'$ are the $d$-dimensional SLEM vectors of $x$ and $x'$, respectively, their cosine similarity reads:
%we can denote $s$ the matching score scalar function defined in $\RR^d\times \RR^d$, which is given by
\begin{equation}
s(\omega, \omega') = \dfrac{\omega^T \omega'}{\|\omega\| \|\omega'\|} = \frac{(x-\mu)^TA^{-2}(x'-\mu)}{\|A^{-1}(x-\mu)\|\|A^{-1}(x'-\mu)\|},\label{match:score}
\end{equation}
which does not depend on the value of $\theta$. This means that the weight of the positive sample can be fixed at any positive value without changing matching scores. This sets the SLEM appart from E-SVM that requires this parameter to be cross validated \cite{Efros11, ZePe15}.
\subsection{LDA and SLEM}

Let us now reanalyse the SLEM problem and suppose that we have multiple positive samples. It can be shown that in this case, the corresponding linear classifier of Equation (\ref{eq:first}) for the square-loss is also given by
(\ref{omega:solution}), where $x_0$ denotes this time the center of mass
of the positive samples, {\em if} these samples have the {\em same} covariance matrix $\Sigma$ as the negative samples $X$.
 
This equal-covariance assumption is of course quite restrictive, and
probably unrealistic in general. It is interesting to note, however,
that this is exactly the assumption made by linear discriminant
analysis. As shown in~\cite{Hastie2009} for example, LDA can be seen as a
(non-regularized) linear classifier
with decision function $\omega'\cdot z+ b'$, where $z$ is a sample in
$\RR^d$, and
\begin{equation}
\left\{\begin{array}{l}
\displaystyle \omega'=\Sigma^{-1}(x_0-\mu),\\
\displaystyle b'=-\frac{1}{2}(x_0+\mu)^T \omega',
\end{array}\right.
\label{eq:lda}
\end{equation}
 
Note that when $\lambda=0$ in SLEM (no regularization),
\begin{align}
\Sigma \omega & = U\omega -\frac{1}{2}[(x_0-\mu)^T \omega] (x_0-\mu) =
\left[1-\frac{1}{2}(x_0-\mu)^T \omega\right](x_0-\mu) \\
&=\left[1-\frac{1}{2}(x_0-\mu)^T \omega\right]\Sigma \omega'
\end{align}
thus $\omega$ and $\omega'$ have the same direction \PP{Not sure to get it. What is $\omega$ here? $\omega^{\star}$? Also it says that $\Sigma \omega$ and $\Sigma \omega'$ are aligned, not $\omega$ and $\omega'$.}. In other words, a
SLEM is a generalized version of LDA when the
regularizer parameter $\lambda $ is zero.
This observation has been equally made by \cite{Koba15}\PP{For square loss e-svm as well?}.

Many interesting properties of LDA has been rediscovered for classification tasks \cite{GMPD12,HMR12} and, more recently, for image retrieval with non-embedded image encoders \cite{babenko15}.



%%% Local Variables:
%%% TeX-master: "main_eccv"
%%% End:

\input{sec_kernel_methods.tex}

\input{sec_eff_imp.tex}

\section{Experimental Evaluation}
\label{eval}

\subsection{Datasets and evaluation protocol} \label{eval:protocol}
Our default dataset in this paper will be INRIA \emph{Holidays} dataset \cite{holidays}. This dataset consists of 1491 images divided in 500 groups of matching images. For the remaining of this report, if the dataset of an experiment is not mentioned, the experiment is performed on \emph{Holidays}.
We also perform in the \emph{Oxford} dataset \cite{oxford}, which consists of 5062 images separated in 55 groups of matching images.

Each group of these datasets contains one query image. 
For each query image, we calculate its similarity to all other images in the database and rank them, in decreasing order. 
The average precision of a group is calculated by the ranking of the images of the group for the similarity with the corresponding query image. 
The final mean average precision (mAP) for a dataset is the mean of the average precision over all its groups.
As negative sample of images, we use a set of images $10^5$ from Flickr \cite{oxford}. 
For a full rank decomposition, we use a subset of Flickr100k, between $6000$ and 15000 images. 
As stated in section \ref{low-rank} and further discussed in section \ref{time-scale}, a full rank decomposition does not scale well for bigger number of negative samples. For low rank decomposition, we use all 100000 images.

At evaluation time, for a dataset that consists of $p$ images and $q$ query images, we calculate its $p\times q$ \emph{similarity matrix} $S$, where each of its $q$ columns is the matching scores of the query image with all the $p$ images.


\subsection{Which kernel to choose?}
We tested four different kernels: linear kernel, RBF kernel, polynomial kernel and spatial pyramid match kernel. Each kernel takes as parameter a scalar $\gamma$.

\begin{align}
    &k_{linear}(x,y) = x^Ty; \label{k:lin}\\
    &k_{rbf}(x,y) = \exp(-\gamma|\! |x-y|\! |^2); \label{k:rbf}\\
    &k_{poly}(x,y) = x^Ty+\gamma(x^Ty)^2; \label{k:poly}\\
    &k_{spp}(x,y) = \dfrac{1}{2^L}\mathcal{I}(H^0_x, H^0_y) +\sum_{l=1}^L\dfrac{1}{2^{L-l+1}}\mathcal{I}(H^{l,_gamma}_x, H^{l,\gamma}_y). \label{k:spm}
\end{align}
\textbf{Linear SLEM} The linear kernel of Equation (\ref{k:lin}) is the first default choice and equivalent to the non-kernelized version of SLEM.
In the remaining of this paper, we reference to linear SLEM when we use the non-kernelized SLEM.
\textbf{Gaussian SLEM} The radial basis function kernel of Equation (\ref{k:rbf}) is a
well known reproducing kernel, used for classification with support vector machines.
\textbf{Polynomial SLEM} The polynomial kernel of Equation (\ref{k:poly}) is a reproducing kernel normally used in natural language processing. 
This three kernels are amount the most used kernels in the machine learning literature.
\textbf{SPM SLEM} The pyramid match kernel of Equation (\ref{k:spm}) is a Mercer kernel introduced in \ref{GrauDa05} 



\subsection{Base visual features}
Local descriptors: RootSIFT \cite{3things} and deep convolutional features \cite{SimonZisser15}

Embedding: VLAD features \cite{VLAD} and aggregation \cite{babenko15}

Caffe CNN features \cite{jia2014caffe}


%We use four base features as the representation in $\RR^d$ of our images. Firstly, we use VLAD features, as used in \cite{ZePe15}. 
%CNN features are non-negative and can be used both as $\mathbb{L}^1$ or $\mathbb{L}^2$ normalized features. The third features are spatial pyramids of SIFT descriptors, as used in \cite{spk}. These are non-negative $\mathbb{L}^1$ normalized features. \emph{\color{red} Details of these features construction are not relevant right now.}


\subsection{Implementation details}
\emph{\color{red} Not relevant right now.  To be written.}
$8192$ dimensional VLAD features.

E-SVM: $0.6$ second to solve one single exemplar.

SLEM: $30$ seconds to solve a $8192\times 8192$ linear system for all exemplars of the dataset.

Kernelized SLEM: at most $0.3$ seconds (for RBF kernel) for each iteration of (\ref{icd:algo}) algorithm plus at most $30$ seconds to solve a $r'\times r'$ system.


\subsection{Full rank results}

\begin{table*}[t]
\begin{center}
\begin{tabular}{|c|c|c|c|c|c|}
\hline
Method, features & VLAD-64 \cite{VLAD}& CNN \cite{jia2014caffe} & Root-SIFT \cite{3things} & SPoC CNN \cite{babenko15} &  SP-CNN \cite{SPPCNN} \\
\hline\hline
Baseline            & 72.7 & 68.2 & 37.4 & 73.1 & 64.7\\
%Whitening           & -    & -    & -    & -    & -\\
LDA                 & 69.4 & 69.2 & -    & 77.5 & -\\
E-SVM               & 77.5 & 71.3 & 40.1 & 79.8 & 64.7 \\
Linear SLEM         & 78   & 72.1 & 38.2 & 78.3 & 68 \\
Gaussian SLEM       & 78.1 & 72.9 & 39.7 & 81.4 & 68.6 \\
Poly SLEM           & 78.1 & 72.9 & -    & 82   &   -   \\
SPM SLEM            & -    & 70.2 & 66.9 & -    & 70.2 \\
\hline
\end{tabular}
\end{center}
\caption{Mean average precision results for INRIA Holidays dataset, expressed as percentages. The - means the tests can not be performed or was not performed yet.}
\end{table*}


\begin{table*}[t]
\begin{center}
\begin{tabular}{|c|c|c|c|c|c|}
\hline
Method, features & VLAD-64 \cite{VLAD}& CNN \cite{jia2014caffe} & SP-SIFT \cite{spk} & SPoC CNN \cite{babenko15} &  SP-CNN \cite{SPPCNN} \\
\hline\hline
Baseline            & 46.3 & 40.6 & - & 54.4 & 44 \\
%Whitening           & 07   & 20.2 & - & -    & -\\
LDA                 & 50.9 & 45   & - & 63.7 & 37.4\\
E-SVM               & 57.5 & 44.6 & - & 62.1 & - \\
Linear SLEM         & 57.5 & 45.5 & - & 64.1 & 45 \\
Gaussian SLEM       & 59   & 46.1 & - & 64.9 & 45 (no cv) \\
Intersection SLEM   & -    & 42.2 & - & -    & - \\
\hline
\end{tabular}
\end{center}
\caption{Mean average precision results for Oxford 5k buildings dataset, expressed as percentages. The - means the tests can not be performed or was not performed yet.}
\end{table*}


\subsection{Time Scalability} \label{time-scale}
\begin{figure}[!h]
\centering
\begin{subfigure}[b]{0.32\textwidth}
\includegraphics[width=\textwidth]{speed_n_6K.png}
\end{subfigure}
\begin{subfigure}[b]{0.32\textwidth}
\includegraphics[width=\textwidth]{speed_n_15K.png}
\end{subfigure}
\begin{subfigure}[b]{0.32\textwidth}
\includegraphics[width=\textwidth]{speed_n.png}
\end{subfigure}
\caption{Comparison between time calculation feature encoding for different methods. At left, we use $n=6000$ negative examples. At right, $n=15000$ negative examples. In both experiments, we use VLAD as base features.}
\label{time:scalar}
\end{figure}

\subsection{Low-rank decomposition evaluation}

\begin{figure}[!h]
\centering
\begin{subfigure}[b]{0.48\textwidth}
\includegraphics[width=\textwidth]{linear_decomposition_nolog.png}
\end{subfigure}
\begin{subfigure}[b]{0.48\textwidth}
\includegraphics[width=\textwidth]{rbf_decomposition_nolog.png}
\end{subfigure}
\caption{Comparison between full rank and low rank. In blue, mAP results for 

At the left, comparison between no kernel results and linear kernel results. In black, mAP of non kernelized SLEM. In red, mAP for different low-rank decompositions of linear kernel matrix. In this experiment, $r=2^{13}$ and we set $\log_2 r'$ in $\{6, 7,...,13\}$. At the right, comparision between the gaussian kernel results and its low-rank decomposition results. In blue, mAP of the full-rank Gaussian SLEM. In red, mAP for different low-rank decomposition of Gaussian SLEM. In this experiment, $r=10281$.}
\label{no.ker.vs.linear2}
\end{figure}

%\caption{Comparison between no kernel results and linear kernel results. In black, mAP of non kernelized SLEM. In red, mAP for different low-rank decompositions of linear kernel matrix. In this experiment, $r=2^{13}$ and we set $\log_2 r'$ in $\{6, 7,...,13\}$. }

%\caption{Comparison between no kernel results and linear kernel results. In black, mAP of non kernelized SLEM. In red, mAP for different low-rank decompositions of linear kernel matrix. In this experiment, $r=2^{13}$ and we set $\log_2 r'$ in $\{6, 7,...,13\}$. }




"Improving bag-of-features for large scale image search", Herv� J�gou, Matthijs Douze, Cordelia Schmid, IJCV 2010
81.3 

"Asymmetric Hamming Embedding", Mihir Jain, Herv� J�gou, Patrick Gros, ACM MM 2011
81.9

"To aggregate or not to aggregate: selective match kernels for image search", Giorgos Tolias, Yannis Avrithis and Herv� J�gou, ICCV 2013
88




\section{Conclusion}
\label{conclusion}
In this paper, we address the problem of large scale image retrieval with global image representation. 
We presented a simples idea, the kernelized square-loss exemplar machine, and yet an elegant implementation. 
As a result, we obtain significant improvement over the image representations we tested, outperform similar encoders and near state of the art performances.
As future work, we must test our method on state of the art representations and see how it impacts their results.
The use of other kernel functions is also encouraged. The polynomial outperforms the Gaussian kernel, even though the Hilbert space obtained from the Gaussian kernel has infinite dimensions \emph{\color{red} fact check}. 
The spatial pyramid kernel can be used to improve representations based on local descriptors


\bibliographystyle{ieee} 
\bibliography{sup,jz}
\end{document}


	
%% Local Variables:
%% TeX-master: "main_eccv"
%% End:
