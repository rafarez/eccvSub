\section{Efficient implementation}\label{eff_imp}

When compared with the problem of (\ref{omega:first}), one drawback of a kernelized approach of (\ref{ker:first}) is that the dimension of our problem grows with the size $n$ of the negative samples.
We can compute the Cholesky factor $B$ of $K$ in $O(nr^2)$ time and $O(nr)$ storage, as will be shown in subsection~\ref{low-rank}. 
For each positive exemplar, solving Equation (\ref{beta:final}) amounts to computing $B'$ from Lemma 1, and solving a linear system in $U$, which is a $(r+1)\times (r+1)$ matrix. The first of these is calculated in time $O(nr)$ and the second, $O(r^3)$.

Two ideas allow us to solve the kernelized problem linearly in $n$: low-rank decomposition of the kernel matrix $K$, to diminish the value of $r$ and the Woodbury identity, so we only have to solve one linear system for all positives. 

\subsection{Low-rank decomposition} \label{low-rank}
In this section we start by revisiting the Cholesky algorithm of the construction of $B$, as presented in ~\cite{BaJo05,FiSc01}. 
We denote, for any list of indexes $M$, $N$ in $\{1,2,...,n\}$, $K(M,N)$ the submatrix of $K$ composed of the rows indexed by $M$ and columns indexed by $N$ and $K(M,:)$ the submatrix of $K$ composed by rows indexed by $M$ and all columns.
Let us assume a sequence of indexes $I_r=\{i_1,i_2,\dots ,i_r\}$ from $\{1,2,\dots, n\}$ is known. 
We also denote, for $t$ in $\{1,2,...,r\}$, $I_t=\{i_1,...,i_t\}$ the subset of the $t$ first indexes of $I_r$ and $J_t=\{1,2,...,n\} \setminus I_t$ the complement of the $t$ first indexes.
$I_r$ is the set of pivots, chosen greedily such that $i_t$ minimizes the error approximation between $B_tB_t^T$ and $K$, where $B_t=B(I_t,:)$. 

Given $I_t$, we can calculate the $t$-th column, $t=1,2,...,r$, of $B$ iteratively:
\begin{align}
\begin{cases}
\vspace{3 mm}
B(i_t, t) = \left(K(i_t,i_t)-\sum_{m=1}^{t-1} B(i_{m},m)  \right)^{\frac{1}{2}},\\
\vspace{3 mm}
B(I_{t-1},t) = 0,\\
B(J_t, t) = \frac{1}{B(i_t,t)}(K(J_t,i_t)  -\sum_{m=1}^{t-1}B(J_t,m)B(i_t,m)
).\end{cases}\label{icd:algo}
\end{align}
The time complexity of the $t$-th interaction is $O(tn)$, thus making the full algorithm $O(nr^2)$ in time. Since $B$ is $n\times r$, storage complexity is $O(nr)$. In particular, \cite{BaJo05} shows us that, for $t$ in $\{1,2,...,r\}$, the error matrix $K-B_tB_t^T$ is a $(n-t)\times (n-t)$ block matrix which is the Schur complement of $K(I_t,I_t)$ in $K$ and zero elsewhere.

For many kernels such as the Gaussian kernel, $K$ can be a full rank matrix. Its Cholesky decomposition is hence a $O(n^3)$ operation in time, which is a prohibitive cost for large negative sets $X$. We propose stopping the algorithm described in (\ref{icd:algo}) after a given number $r'<r$ of steps, and use $B_{r'}$ as an approximation of $B$. We also need to know only the $r'$ first pivots, which we call $I$ from now on. Conversely, we note $J=J_{r'}$. The construction of $B'$ from $B_{r'}$ can be adapted as presented in the following lemma:
%\begin{proposition}
%There is an unique $n\times n$ matrix $L$ such that
%\begin{itemize}
%\item L is symmetric and positive semidefinite,
%\item L(:,I) = K(:,I),
%\item the column space of $L$ is equal to the column space of $L(:,I)$.
%\end{itemize}
%L is such that 
%\begin{equation}
%L([I \ J],[I\ J]) = \begin{bmatrix}
%K(I,I) && K(J,I)^T\\ K(J,I) && K(J,I)K(I,I)^{-1}K(J,I)^T
%\end{bmatrix}.
%\end{equation}
%\end{proposition}
%From \textbf{Proposition 1}, $L$ is an unique approximation of $K$ for a given pair $I$ and $J$. The algorithm from (\ref{icd:algo}) construct $B$ such that $L=BB^T$.
\begin{lemma}
The augmented kernel matrix $K'$ can be factorized as $K'\approx B'B'^T$ where
\begin{equation}
\begin{split}
& B'=\begin{bmatrix} u & v^T\\w & B_{r'} \end{bmatrix},~ v=B_I^{-1} k_0(I),~ u=\sqrt{k_{00}-\|v\|^2},\\
& w(I) = 0,~w(J)=\dfrac{1}{u}(k_0(J)-B_Jv),
\end{split}
\end{equation}
in time $O(nr')$ and storage $O(n)$. Also, the approximation error of $K'$ and $B'B'^T$ is the same of $K$ and $B_{r'}B_{r'}^T$. Here, $B_I$ and $B_J$ denote $B(I,I)$ and $B(J,I)$, respectively.
\end{lemma}
The proof of this lemma is similar to the previous lemma's proof, only adding the $\RR^n$ vector $w$ whose entries are calculated by Equation (\ref{icd:algo}).
%\begin{proof}
%As before, we have that
%\begin{equation}
%B'B'^T = \begin{bmatrix} u+|\!|v|\!|^2 & v^TB_{r'}^T+uw^T\\ 
%B_{r'}v+uw& BB^T+ww^T\end{bmatrix}.
%\end{equation}
%If $E = K'-B'B'^T$,
%Algorithm (\ref{icd:algo}) is such that the rank of $B_{r'}$ is $r'$, for all $r'<r$.
%\end{proof}
\subsection{Woodbury identity}
Let us call $A=\frac{1}{n}QQ^T-\mu'\mu'^T+\lambda\textbf{Id}_{r+1}$, where $Q$ and $\mu'$ are defined at the end of Section \ref{subsec:adding}. We assume its inverse $A^{-1}$ known. From the kernalized version of (\ref{omega:solution}), we know that $U=A+\frac{\theta}{\theta+1}\delta\delta^T$, with $\delta=b_0'-\mu'$. The Woodbury identity gives us 
\begin{equation}
U^{-1}=A^{-1} -\dfrac{\theta}{\theta\delta^TA^{-1}\delta+ \theta+1}A^{-1}\delta^T\delta A^{-1}.\label{wood1}
\end{equation}
Replacing (\ref{wood1}) in the original equation yields the solution
\begin{equation}
\beta^\star = \dfrac{2\theta}{\theta+1}U^{-1}\delta = \dfrac{2\theta}{\theta\delta^TA^{-1}\delta+ \theta+1}A^{-1}\delta,\label{beta:fromwood}
\end{equation}
which depends only $A^{-1}$, $\theta$, $\mu'$ and $b_0'$. 
We can thus solve only one linear system with matrix $A$ for \textit{all} positives, instead of one in $U$ for \textit{each} positive.

If we decompose $K$ at rank $r$, $A$ is constant for all positives, since $Q = [0, B]$ and $\mu'=[0; \mu]$.\footnote{Here, $\mu$ is the center of mass of the rows of $B$, not the columns of $X$ as initially presented in Section \ref{lsesvm}.} 
But for a low-rank decomposition, we have instead $Q=[w, B_{r'}]$ and $\mu' = [\bar{w}; \mu]$. If so, $A$ is written as 
\begin{equation}
A = \begin{bmatrix}
a_{00} & a_0^T\\
a_0^T & G
\end{bmatrix},
\end{equation}
where $a_{00}=\dfrac{1}{n}w^Tw-\bar{w}^2+\lambda$, $a_0 = \dfrac{1}{n}B_{r'}^Tw-\bar{w}\mu$ and $G=\dfrac{1}{n}B_{r'}^TB_{r'}-\mu\mu^T+\lambda\text{Id}_r$.

Matrix $G$ depends only on $B$ and can be calculated in time $O(nr'^2)$ and storage $O(r'^2)$. 
Its inverse $G^{-1}$ is computed at cost $O(r'^3)$.
The Woodbury identity, again, allows the computation of $A^{-1}$ from $G^{-1}$ at cost $O(r^2)$:
\begin{equation}
A^{-1} = \begin{bmatrix}\gamma & -\gamma a_0^TG^{-1}\\ -\gamma G^{-1}a_0 & G^{-1}+\gamma G^{-1}a_0a_0^TG^{-1}\end{bmatrix},\label{wood2}
\end{equation}
where $\gamma = \left(a_{00}-a_0^TG^{-1}a_0\right)^{-1}$. Finally, replacing (\ref{wood2}) in (\ref{beta:fromwood}), we can obtain $\beta^\star$ by solving the linear problem $A\beta^\star = \delta$. 
Both $A$ and $\delta$ are dependent on $u$, $v$ and $w$, which are computed for each exemplar as described by Lemma 2.

%%% Local Variables:
%%% TeX-master: "main_eccv"
%%% End: