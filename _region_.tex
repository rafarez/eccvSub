\message{ !name(main_eccv.tex)}\documentclass[runningheads]{llncs}

%\usepackage{times}
%\usepackage{epsfig}
\usepackage{graphicx}
\usepackage{caption}
%\usepackage{subfigure}
\usepackage{subcaption}
\captionsetup{compatibility=false}
\usepackage{amsmath}
\usepackage{amssymb}
\usepackage{ruler}
\usepackage{color}
%\usepackage{pgfplots}
\usepackage[width=122mm,left=12mm,paperwidth=146mm,height=193mm,top=12mm,paperheight=217mm]{geometry}
%\newtheorem{proposition}{Proposition}
%\newtheorem{lemma}{Lemma}
%\newtheorem{proof}{Proof}
\newcommand{\RR}{\mathbb R}
\newcommand{\NNN}{\mathcal N}
\newcommand{\sumi}{\displaystyle{\sum_{i=1}^n}}
\DeclareMathOperator*{\argmin}{argmin}

% Highlighting
\usepackage{soul}
\usepackage{color}
%\usepackage[usenames,dvipsnames]{xcolor}
\newcommand{\hlc}[2][yellow]{{\sethlcolor{#1}\hl{#2}}}
\newcommand{\RAF}[1]{\hlc[yellow]{(RR:) #1}}
\newcommand{\JZ}[1]{\hlc[pink]{(JZ:) #1}}
\newcommand{\PP}[1]{\hlc[green]{PP: #1}}

\usepackage{tikz}
\usepackage{pgfplots,pgfplotstable}
\usetikzlibrary{pgfplots.groupplots}
\pgfplotsset{grid=major,height=2in, width=\columnwidth}
\tikzset{PolySLEM/.style={mark=*, blue}}
\tikzset{GaussSLEM/.style={mark=o, green}}
\tikzset{ESVM/.style={mark=x, red}}

%\tikzset{ussvmne/.style={mark=None, blue, dash pattern=on 4pt off 1pt on 1pt off 1pt}}
%\tikzset{l1svm/.style={mark=*, red, mark options={solid, fill=red!80!black}, solid}}
%\tikzset{l2svm/.style={mark=*, brown!60!black, mark options={solid, fill=brown!80!black}, solid}}


\pgfplotsset{compat=newest,legend columns=-1}

\pgfplotstableread{
Criterion   criterion1  criterion2
part1       8           2349
part2       8           452
part3       14          1006
}\first

\pgfplotstableread{
Criterion   criterion1  criterion2
part1       36          10220
part2       36          5891
part3       85          3160
}\second

\pgfplotstableread{
Criterion   criterion1  criterion2
part1       97          25657
part2       97          18306
part3       184         7461
}\third

\begin{document}

\message{ !name(figuren.tex) !offset(-43) }
\begin{figure}
  \begin{tikzpicture}
    \begin{groupplot}
      [group style={%
        columns=3,
        group name=plots,
        xlabels at=edge bottom,
        y descriptions at=edge left,
      },
      ybar,
      ymin=0,
      ymax=27e3,
      enlarge x limits={abs=.5},
      width=0.35\textwidth,
      height=0.6\textwidth,
      scaled y ticks=base 10:-3,
      xticklabels from table={\first}{Criterion},
      x tick label style={rotate=90,anchor=east},
      xtick=data,
      ]

      \nextgroupplot[xlabel=item1,legend to name=grouplegend,ylabel=y-label]
      \pgfplotstableforeachcolumn\first\as\col{%
        \ifnum\pgfplotstablecol=0
        \else
        \edef\tmp{%
          \noexpand\addplot table [x expr=\noexpand\coordindex,y=\col] {\noexpand\first};
          \noexpand\addlegendentry {\col}%
        }%
        \tmp
        \fi
      }

      \nextgroupplot[xlabel=item2]
      \pgfplotstableforeachcolumn\second\as\col{%
        \ifnum\pgfplotstablecol=0 
        \else
        \edef\tmp{%
          \noexpand\addplot table [x expr=\noexpand\coordindex,y=\col] {\noexpand\second};
        }%
        \tmp
        \fi
      }

      \nextgroupplot[xlabel=item3]
      \pgfplotstableforeachcolumn\third\as\col{%
        \ifnum\pgfplotstablecol=0 
        \else
        \edef\tmp{%
          \noexpand\addplot table [x expr=\noexpand\coordindex,y=\col] {\noexpand\third};
        }%
        \tmp
        \fi
      }
    \end{groupplot}

    \node at (plots c2r1.north) [anchor=south, yshift=.6cm] {\ref{grouplegend}};
  \end{tikzpicture}
\end{figure}


% \def\sideboxwidth{0.48\textwidth}
% \def\centerboxwidth{0.02\textwidth}
% \begin{figure}[!h]
%   \begin{minipage}{\sideboxwidth}
%     \begin{center}
%       \begin{tikzpicture}
% 	\begin{axis}[
%           xlabel=Num. of negatives,
%           ylabel=mAP]
%           %% Poly SLEM
%           \addplot[PolySLEM] coordinates {
%             (500,  0.78852)
%             (1500, 0.80397)
%             (2500, 0.80648)
%             (3500, 0.80905)
%             (4500, 0.81192)
%             (5500, 0.81502)
%             (6500, 0.81659)
%             (7500, 0.81712)
%             (8500, 0.81599)
%             (9500, 0.81432)
%             (10500,0.81625)
%             (11500,0.81796)
%             (12500,0.81726)
%             (13500,0.81993)
%             (14500,0.82057)
%           };
%           %% Gaussian SLEM
%           \addplot[GaussSLEM] coordinates {
%             (500,  0.77757)
%             (1500, 0.79756)
%             (2500, 0.80338)
%             (3500, 0.79909)
%             (4500, 0.79950)
%             (5500, 0.80257)
%             (6500, 0.8016)
%             (7500, 0.80362)
%             (8500, 0.80377)
%             (9500, 0.80501)
%             (10500,0.80534)
%             (11500,0.80909)
%             (12500,0.80955)
%             (13500,0.80813)
%             (14500,0.81214)
%           };
%           %% ESVM
%           \addplot[ESVM] coordinates {
%             (500,  0.7684)
%             (1500, 0.78434)
%             (2500, 0.78836)
%             (3500, 0.78927)
%             (4500, 0.79292)
%             (5500, 0.79523)
%             (6500, 0.79491)
%             (7500, 0.79557)
%             (8500, 0.79915)
%             (9500, 0.79819)
%             (10500,0.79888)
%             (11500,0.7985)
%             (12500,0.79822)
%             (13500,0.79930)
%             (14500,0.79905)
%           };
%           %% Linear SLEM
%           \addplot[LinSLEM] coordinates {
%             (500,  0.77638)
%             (1500, 0.7748)
%             (2500, 0.77919)
%             (3500, 0.77914)
%             (4500, 0.7748)
%             (5500, 0.77578)
%             (6500, 0.77902)
%             (7500, 0.78338)
%             (8500, 0.78043)
%             (9500, 0.78026)
%             (10500,0.77803)
%             (11500,0.77767)
%             (12500,0.77818)
%             (13500,0.78148)
%             (14500,0.78254)
%           };
% 	\end{axis}
%       \end{tikzpicture}
%     \end{center}
%   \end{minipage}%
%   \begin{minipage}{\centerboxwidth}~\end{minipage}%
%   \begin{minipage}{\sideboxwidth}
%     \begin{center}
%       \begin{tikzpicture}
% 	\begin{semilogyaxis}[
%           xlabel=Num. of negatives,
%           ylabel=Time per image (s),
%           legend style={at={(0.465,-0.45)},
%             anchor=north,legend columns=-1},
%           ]%legend pos=outer south]
%           %% Poly SLEM
%           \addplot[PolySLEM] coordinates {
%             (500,  0.0001)
%             (1500, 0.0007)
%             (2500, 0.0023)
%             (3500, 0.0046)
%             (4500, 0.0088)
%             (5500, 0.0146)
%             (6500, 0.0213)
%             (7500, 0.0289)
%             (8500, 0.0395)
%             (9500, 0.0542)
%             (10500,0.0703)
%             (11500,0.0903)
%             (12500,0.1103)
%             (13500,0.1434)
%             (14500,0.1745)
%           };
%           \addlegendentry{Poly SLEM}
%           %% Gaussian SLEM
%           \addplot[GaussSLEM] coordinates {
%             (500,  0.0003)
%             (1500, 0.0008)
%             (2500, 0.0021)
%             (3500, 0.0044)
%             (4500, 0.0072)
%             (5500, 0.0119)
%             (6500, 0.0172)
%             (7500, 0.026)
%             (8500, 0.0344)
%             (9500, 0.0465)
%             (10500,0.0617)
%             (11500,0.0814)
%             (12500,0.1248)
%             (13500,0.1301)
%             (14500,0.161)
%           };
%           \addlegendentry{Gaussian SLEM}
%           %% ESVM
%           \addplot[ESVM] coordinates {
%             (500,  0.014)
%             (1500, 0.015)
%             (2500, 0.016)
%             (3500, 0.017)
%             (4500, 0.018)
%             (5500, 0.019)
%             (6500, 0.02)
%             (7500, 0.021)
%             (8500, 0.022)
%             (9500, 0.023)
%             (10500,0.024)
%             (11500,0.025)
%             (12500,0.026)
%             (13500,0.027)
%             (14500,0.028)
%           };
%           \addlegendentry{ESVM}
%           %% Linear SLEM
%           \addplot[LinSLEM] coordinates {
%             (500,  0.0000565)
%             (1500, 0.0000575)
%             (2500, 0.0000467)
%             (3500, 0.0000679)
%             (4500, 0.0000683)
%             (5500, 0.0000612)
%             (6500, 0.0000608)
%             (7500, 0.0000708)
%             (8500, 0.0000743)
%             (9500, 0.0000996)
%             (10500, 0.0000993)
%             (11500,0.0001073)
%             (12500,0.0001027)
%             (13500,0.000075)
%             (14500,0.0001141)
%           };
%           \addlegendentry{Linear SLEM}
% 	\end{semilogyaxis}
%       \end{tikzpicture}
%       \caption{Results for INRIA Holidays, using SPoC features and different methods of SLEM (see legend).}
%       \label{fullrank:results}
%     \end{center}
%   \end{minipage}
% \end{figure}

%% Local Variables:
%% TeX-master: "main_eccv"
%% End:

\message{ !name(main_eccv.tex) !offset(-205) }

\end{document}


	
%%% Local Variables:
%%% TeX-master: "main_eccv"
%%% End:
