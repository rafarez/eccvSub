% \begin{figure}
%   \begin{tikzpicture}
%     \begin{groupplot}
%       [group style={%
%         columns=2,
%         rows=1,
%         group name=plots,
%         xlabels at=edge bottom,
%         %y descriptions at=all,
%         horizontal sep=5em,        
%       },
%       % ybar,
%       % ymin=0,
%       % ymax=27e3,
%       enlarge x limits={abs=.5},
%       width=0.5\textwidth,
%       height=0.4\textwidth,
%       % scaled y ticks=base 10:-3,
%       % xticklabels from table={\first}{Criterion},
%       % x tick label style={rotate=90,anchor=east},
%       % xtick=data,
%       ]

%       \nextgroupplot[xlabel=Num. of negatives $n$,
%       ylabel=mAP]
%       ]
%       %% Poly SLEM
%       \addplot[PolySLEM] coordinates {
%         (500,  0.78852)
%         (1500, 0.80397)
%         (2500, 0.80648)
%         (3500, 0.80905)
%         (4500, 0.81192)
%         (5500, 0.81502)
%         (6500, 0.81659)
%         (7500, 0.81712)
%         (8500, 0.81599)
%         (9500, 0.81432)
%         (10500,0.81625)
%         (11500,0.81796)
%         (12500,0.81726)
%         (13500,0.81993)
%         (14500,0.82057)
%       };
%       %% Gaussian SLEM
%       \addplot[GaussSLEM] coordinates {
%         (500,  0.77757)
%         (1500, 0.79756)
%         (2500, 0.80338)
%         (3500, 0.79909)
%         (4500, 0.79950)
%         (5500, 0.80257)
%         (6500, 0.8016)
%         (7500, 0.80362)
%         (8500, 0.80377)
%         (9500, 0.80501)
%         (10500,0.80534)
%         (11500,0.80909)
%         (12500,0.80955)
%         (13500,0.80813)
%         (14500,0.81214)
%       };
%       %% ESVM
%       \addplot[ESVM] coordinates {
%         (500,  0.7684)
%         (1500, 0.78434)
%         (2500, 0.78836)
%         (3500, 0.78927)
%         (4500, 0.79292)
%         (5500, 0.79523)
%         (6500, 0.79491)
%         (7500, 0.79557)
%         (8500, 0.79915)
%         (9500, 0.79819)
%         (10500,0.79888)
%         (11500,0.7985)
%         (12500,0.79822)
%         (13500,0.79930)
%         (14500,0.79905)
%       };
%       %% Linear SLEM
%       \addplot[LinSLEM] coordinates {
%         (500,  0.77638)
%         (1500, 0.7748)
%         (2500, 0.77919)
%         (3500, 0.77914)
%         (4500, 0.7748)
%         (5500, 0.77578)
%         (6500, 0.77902)
%         (7500, 0.78338)
%         (8500, 0.78043)
%         (9500, 0.78026)
%         (10500,0.77803)
%         (11500,0.77767)
%         (12500,0.77818)
%         (13500,0.78148)
%         (14500,0.78254)
%       };

%       \nextgroupplot[
%       ymode=log,
%       xlabel=Num. of negatives $n$,
%       ylabel=Time per image (s),
%       legend to name=grouplegend,
%       legend style={legend columns=-1},
%       % legend style={at={(0.465,-0.45)},
%       % anchor=north,legend columns=-1},
%       ]%
%       %% Poly SLEM
%       \addplot[PolySLEM] coordinates {
%         (500,  0.0001)
%         (1500, 0.0007)
%         (2500, 0.0023)
%         (3500, 0.0046)
%         (4500, 0.0088)
%         (5500, 0.0146)
%         (6500, 0.0213)
%         (7500, 0.0289)
%         (8500, 0.0395)
%         (9500, 0.0542)
%         (10500,0.0703)
%         (11500,0.0903)
%         (12500,0.1103)
%         (13500,0.1434)
%         (14500,0.1745)
%       };
%       \addlegendentry{Poly SLEM}
%       %% Gaussian SLEM
%       \addplot[GaussSLEM] coordinates {
%         (500,  0.0003)
%         (1500, 0.0008)
%         (2500, 0.0021)
%         (3500, 0.0044)
%         (4500, 0.0072)
%         (5500, 0.0119)
%         (6500, 0.0172)
%         (7500, 0.026)
%         (8500, 0.0344)
%         (9500, 0.0465)
%         (10500,0.0617)
%         (11500,0.0814)
%         (12500,0.1248)
%         (13500,0.1301)
%         (14500,0.161)
%       };
%       \addlegendentry{Gaussian SLEM}
%       %% ESVM
%       \addplot[ESVM] coordinates {
%         % (600, .000610)
%         % (6000, .00427)
%         % (60000, .0378)
%         (1000 * 6/10, .000610)
%         (10000* 6/10, .00427)
%         (14500, .0095478) %Interpolated
%         (100000 * 6/10, .0378)
%         % (500,  0.014)
%         % (1500, 0.015)
%         % (2500, 0.016)
%         % (3500, 0.017)
%         % (4500, 0.018)
%         % (5500, 0.019)
%         % (6500, 0.02)
%         % (7500, 0.021)
%         % (8500, 0.022)
%         % (9500, 0.023)
%         % (10500,0.024)
%         % (11500,0.025)
%         % (12500,0.026)
%         % (13500,0.027)
%         % (14500,0.028)
%       };
%       \addlegendentry{ESVM}
%       %% Linear SLEM
%       \addplot[LinSLEM] coordinates {
%         (500,  0.0000565)
%         (1500, 0.0000575)
%         (2500, 0.0000467)
%         (3500, 0.0000679)
%         (4500, 0.0000683)
%         (5500, 0.0000612)
%         (6500, 0.0000608)
%         (7500, 0.0000708)
%         (8500, 0.0000743)
%         (9500, 0.0000996)
%         (10500, 0.0000993)
%         (11500,0.0001073)
%         (12500,0.0001027)
%         (13500,0.000075)
%         (14500,0.0001141)
%       };
%       \addlegendentry{Linear SLEM}
      


%     \end{groupplot}

%     \node at (plots c1r1.north east) [anchor=south, xshift=2.5em] {\ref{grouplegend}};
%     %\draw (plots c2r1.north west) circle (3pt) node {North west};

%   \end{tikzpicture}
%   \caption{Results for INRIA Holidays, using SPoC features and different methods of SLEM (see legend).We use $T=10^5$ iterations for all $n$ to report mAP for ESVM, as per \cite{ZePe15}, but report timings using $T=1.66 n$ and the values reported in Table 1 of \cite{ZePe15}. Left: mAP; right: online computation time.}
%   \label{fullrank:results}
% \end{figure}


% \begin{figure}
%   \begin{tikzpicture}
%     \begin{groupplot}
%       [group style={%
%         columns=3,
%         group name=plots,
%         xlabels at=edge bottom,
%         y descriptions at=edge left,
%       },
%       ybar,
%       ymin=0,
%       ymax=27e3,
%       enlarge x limits={abs=.5},
%       width=0.35\textwidth,
%       height=0.6\textwidth,
%       scaled y ticks=base 10:-3,
%       xticklabels from table={\first}{Criterion},
%       x tick label style={rotate=90,anchor=east},
%       xtick=data,
%       ]

%       \nextgroupplot[xlabel=item1,legend to name=grouplegend,ylabel=y-label]
%       \pgfplotstableforeachcolumn\first\as\col{%
%         \ifnum\pgfplotstablecol=0
%         \else
%         \edef\tmp{%
%           \noexpand\addplot table [x expr=\noexpand\coordindex,y=\col] {\noexpand\first};
%           \noexpand\addlegendentry {\col}%
%         }%
%         \tmp
%         \fi
%       }

%       \nextgroupplot[xlabel=item2]
%       \pgfplotstableforeachcolumn\second\as\col{%
%         \ifnum\pgfplotstablecol=0 
%         \else
%         \edef\tmp{%
%           \noexpand\addplot table [x expr=\noexpand\coordindex,y=\col] {\noexpand\second};
%         }%
%         \tmp
%         \fi
%       }

%       \nextgroupplot[xlabel=item3]
%       \pgfplotstableforeachcolumn\third\as\col{%
%         \ifnum\pgfplotstablecol=0 
%         \else
%         \edef\tmp{%
%           \noexpand\addplot table [x expr=\noexpand\coordindex,y=\col] {\noexpand\third};
%         }%
%         \tmp
%         \fi
%       }
%     \end{groupplot}

%     \node at (plots c2r1.north) [anchor=south, yshift=.6cm] {\ref{grouplegend}};
%   \end{tikzpicture}
% \end{figure}


\begin{figure}[!ht]
\begin{center}
  \begin{tikzpicture}
    \begin{axis}[
          xlabel=Num. of negatives,          
          ylabel=mAP]
          %% Poly SLEM
          \addplot[PolySLEM] coordinates {
            (500,  0.78852)
            (1500, 0.80397)
            (2500, 0.80648)
            (3500, 0.80905)
            (4500, 0.81192)
            (5500, 0.81502)
            (6500, 0.81659)
            (7500, 0.81712)
            (8500, 0.81599)
            (9500, 0.81432)
            (10500,0.81625)
            (11500,0.81796)
            (12500,0.81726)
            (13500,0.81993)
            (14500,0.82057)
          };
          %% Gaussian SLEM
          \addplot[GaussSLEM] coordinates {
            (500,  0.77757)
            (1500, 0.79756)
            (2500, 0.80338)
            (3500, 0.79909)
            (4500, 0.79950)
            (5500, 0.80257)
            (6500, 0.8016)
            (7500, 0.80362)
            (8500, 0.80377)
            (9500, 0.80501)
            (10500,0.80534)
            (11500,0.80909)
            (12500,0.80955)
            (13500,0.80813)
            (14500,0.81214)
          };
          %% ESVM
          \addplot[ESVM] coordinates {
            (500,  0.7684)
            (1500, 0.78434)
            (2500, 0.78836)
            (3500, 0.78927)
            (4500, 0.79292)
            (5500, 0.79523)
            (6500, 0.79491)
            (7500, 0.79557)
            (8500, 0.79915)
            (9500, 0.79819)
            (10500,0.79888)
            (11500,0.7985)
            (12500,0.79822)
            (13500,0.79930)
            (14500,0.79905)
          };
          %% Linear SLEM
          \addplot[LinSLEM] coordinates {
            (500,  0.77638)
            (1500, 0.7748)
            (2500, 0.77919)
            (3500, 0.77914)
            (4500, 0.7748)
            (5500, 0.77578)
            (6500, 0.77902)
            (7500, 0.78338)
            (8500, 0.78043)
            (9500, 0.78026)
            (10500,0.77803)
            (11500,0.77767)
            (12500,0.77818)
            (13500,0.78148)
            (14500,0.78254)
          };
	\end{axis}
      \end{tikzpicture}
    
      \begin{tikzpicture}
	\begin{semilogyaxis}[
          xlabel=Num. of negatives,
          ylabel=Time per image (s),          
          legend style={at={(0.465,-0.45)},
            anchor=north,legend columns=-1},
          ]%legend pos=outer south]
          %% Poly SLEM
          \addplot[PolySLEM] coordinates {
            (500,  0.0001)
            (1500, 0.0007)
            (2500, 0.0023)
            (3500, 0.0046)
            (4500, 0.0088)
            (5500, 0.0146)
            (6500, 0.0213)
            (7500, 0.0289)
            (8500, 0.0395)
            (9500, 0.0542)
            (10500,0.0703)
            (11500,0.0903)
            (12500,0.1103)
            (13500,0.1434)
            (14500,0.1745)
          };
          \addlegendentry{Poly SLEM}
          %% Gaussian SLEM
          \addplot[GaussSLEM] coordinates {
            (500,  0.0003)
            (1500, 0.0008)
            (2500, 0.0021)
            (3500, 0.0044)
            (4500, 0.0072)
            (5500, 0.0119)
            (6500, 0.0172)
            (7500, 0.026)
            (8500, 0.0344)
            (9500, 0.0465)
            (10500,0.0617)
            (11500,0.0814)
            (12500,0.1248)
            (13500,0.1301)
            (14500,0.161)
          };
          \addlegendentry{Gaussian SLEM}
          %% ESVM
          \addplot[ESVM] coordinates {
            (1000 * 6/10, .000610)
            (10000* 6/10, .00427)            
            (14500, .0095478) %Interpolated
            % (100000 * 6/10, .0378)
            % (500,  0.014)
            % (1500, 0.015)
            % (2500, 0.016)
            % (3500, 0.017)
            % (4500, 0.018)
            % (5500, 0.019)
            % (6500, 0.02)
            % (7500, 0.021)
            % (8500, 0.022)
            % (9500, 0.023)
            % (10500,0.024)
            % (11500,0.025)
            % (12500,0.026)
            % (13500,0.027)
            % (14500,0.028)
          };
          \addlegendentry{ESVM}
          %% Linear SLEM
          \addplot[LinSLEM] coordinates {
            (500,  0.0000565)
            (1500, 0.0000575)
            (2500, 0.0000467)
            (3500, 0.0000679)
            (4500, 0.0000683)
            (5500, 0.0000612)
            (6500, 0.0000608)
            (7500, 0.0000708)
            (8500, 0.0000743)
            (9500, 0.0000996)
            (10500, 0.0000993)
            (11500,0.0001073)
            (12500,0.0001027)
            (13500,0.000075)
            (14500,0.0001141)
          };
          \addlegendentry{Linear SLEM}
	\end{semilogyaxis}
      \end{tikzpicture}
      \caption{Results for INRIA Holidays, using SPoC features and different methods of SLEM (see legend).We use $T=10^5$ iterations for all $n$ to report mAP for ESVM, as per \cite{ZePe15}, but report timings using $T=1.66 n$ and the values reported in Table 1 of \cite{ZePe15}. Left: mAP; right: online computation time.}
      \label{fullrank:results}
    \end{center}
\end{figure}

%% Local Variables:
%% TeX-master: "main_eccv"
%% End:
