\subsection{Comparison with the state of the art}
\label{sec:sota}
\begin{table}[!t]
  \centering
  \caption{Comparison to state-of-the-art.}
  \label{tbl:sota}
  %\begin{tabular}{|c|c|c|c|c|}
  \setlength{\tabcolsep}{.5em}
  \begin{tabular}{ccccc}
    \toprule
    Dataset & dim &\textbf{Holidays} & \textbf{Oxford 5k} & \textbf{Oxford 105k}\\
    \midrule
    SPoC+GaussianSLEM                           & 1e3  &  81.4           &  \textbf{64.9}  &   -  \\
    SPoC+PCA+PolySLEM                           & 1e3  &  \textbf{82.0}    &   64.5  &  \textbf{62.3} \\
    Babenko \textit{et al.} \cite{babenko15}     & 2e2  &  80.2           &   58.9  &  57.8 \\
    J\'egou \textit{et al.} \cite{sota1}         & -    &  81.3           &    -    &   -   \\
    Jain \textit{et al. } \cite{JaJeGro11}       &   -  &  81.9           &    -    &   -   \\
    \midrule
    Tolias \textit{et al.} \cite{Tolias13}      & 1e7  &  \textbf{88.0}     &  \textbf{82.0}    &  \textbf{75.0}   \\
    \bottomrule
  \end{tabular}
  \label{tab:sota}
\end{table}

In Table \ref{tbl:sota} we compare our proposed SLEM architectures to various state-of-the-art methods. Note that our method outperforms all methods except the very-high dimensional method of Toilas \textit{et al.} \cite{Tolias13} which uses features of size $1e7$.


%% Local Variables:
%% TeX-master: "main_eccv"
%% End:
