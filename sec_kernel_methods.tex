\section{Kernel methods for SLEM}
\label{nonlinear SLEM}

Let us recall a few basic facts about kernel methods for supervised
classification. We consider a reproducing kernel Hilbert space (RKHS)
$H$ formed by real functions over some set
$X$, and denote by $k$ the corresponding reproducing kernel. Space $H$ is endowed with an inner product $\langle \cdot, \cdot \rangle$ and associated norm $\|\cdot \|$.

We address the following learning problem over $H\times\RR$:
\begin{equation}
\min_{h\in H,\nu\in\RR}
\sum_{i=1}^n l(y_i,h(x_i)+\nu) + \frac{\lambda}{2}\|h\|^2,
\label{eq:kernel}
\end{equation}  
where the pairs $(x_i,y_i)\in X\times \{-1,1\}$, $i=1\dots n$ are training samples, and $l: \{-1,1\}\times \RR\rightarrow\RR^+$ is some arbitrary loss function. 
By definition of a reproducing kernel,
Equation (\ref{eq:kernel}) can be rewritten as
\begin{equation}
\min_{h\in H,\nu\in\RR}
\sum_{i=1}^n l(y_i,\langle \varphi(x_i),h\rangle+\nu) +
\dfrac{\lambda}{2}\|h\|^2,
\label{ker:aff}
\end{equation} 
where $\varphi$ is the {\em feature map} over $X$ associated with the
kernel $k$ (which may not admit a known explicit form). %and $\langle \cdot, \cdot \rangle$ is the inner product of $H$. 
We dub problems with the general form of (\ref{ker:aff}) {\em affine}
supervised learning problems since, given some fixed element $h$ of
$H$ and some scalar $\nu$, $\langle h,h'\rangle+\nu$ is an affine function of $h'$,
whose zero set defines an affine hyperplane of $H$ considered itself
as an affine space.

Let $K$ denote the kernel matrix with entries $k_{ij}=\langle\varphi(x_i),
\varphi(x_j)\rangle$ and rows $k_i^T=[k_{i1}, k_{i2},...,k_{in}]$, $i$ in $\{1,\ldots,n\}$.  We assume from now on that $l$ is convex and continuous. Under this assumption, Equation (\ref{ker:aff}) admits an equivalent formulation
\begin{align}
\min_{\alpha\in\RR^{n},\nu\in\RR} \left(\dfrac{1}{n}\sumi l(y_i, k_i^T\alpha+\nu)  +\dfrac{\lambda}{2}\alpha^TK\alpha\right), \label{ker:first}
\end{align}
and any solution $(\alpha^\star,\nu^\star)$ to (\ref{ker:first})
provides
a solution $(h^\star,\nu^\star)$ to (\ref{ker:aff}) with
$h^\star=\sum_{i=1}^n \alpha_i^\star\varphi(x_i)+\nu^\star$. This result follows from the representer theorem
~\cite{SHS01,Wahba90}.


%where $K$ is a $(n+1)\times (n+1)$  matrix and $k_i$ is its $(i+1)$-th column matrix for $i$ in $\{0, 1, ..., n\}$.\\

Let us assume from now on that $K$ is a semidefinite positive matrix
%\footnote{\textit{i.e.} for any $z$ in $\RR^n$, $z^TKz\ge 0$. This is equivalent to all eigenvalues of $K$ being non-negative.} 
of rank $r$. Its Cholesky decomposition reads $K=BB^T$, with $B$ a rank $r$ upper triangular matrix of size $n\times r$. Using this decomposition, the kernelized problem can be rewritten as 
\begin{equation}
\min_{\beta\in\RR^r,\nu\in\RR} \left( \dfrac{1}{n}\sumi l(y_i, b_i^T\beta+\nu)+\dfrac{\lambda}{2}\|\beta\|^2\right), \label{beta:first}
\end{equation}
where $b_i^T$ denotes the $i$-th row of $B$.

\subsection{Adding a positive exemplar}
\label{subsec:adding}
Let us reconsider the exemplar problem, with one positive example $x_0$ and $n$ negative examples $X$. Here $K$ is the kernel matrix in $\RR^{n\times n}$ of the negative samples $X$. 
Let us also assume its Cholesky factorization $K=BB^T$ is given as well as the pseudoinverse $B^\dagger$ of its factor. We further analyse how to compute $B$ in Subsection~\ref{low-rank}.

We denote $K'$ the augmented kernel matrix obtained by adding the positive samples $x_0$. Such matrix can be written as
\begin{equation}
K' = \begin{bmatrix}
k_{00} & k_0^T\\
k_0 & K
\end{bmatrix},
\end{equation}
where $k_{00}=\langle \varphi(x_0),\varphi(x_0)\rangle$ is a scalar and $k_0= [\langle \varphi(x_0),\varphi(x_i)\rangle]_{1\le i\le n}$ is a vector in $\RR^n$. 
The following lemma shows that the rank $r+1$ Cholesky factorization of $K'$ can be derived from the Cholesky factorization of its sub-matrix $K$.

\begin{lemma} The augmented kernel matrix $K'$ can be factorized as $K'= B'B'^T$ with
%\begin{align}
%K'&= B'B'^T,\quad\text{where}\quad
%B'=\begin{bmatrix}
%u & v^T\\0 & B
%\end{bmatrix}\\
%v&=B^\dagger k_0,\,\, u=\sqrt{k_{00}-||v||^2}.\label{eq:lemma1}
%\end{align}
\begin{equation}
B'=\begin{bmatrix}
u & v^T\\0 & B
\end{bmatrix},~
v = B^\dagger k_0,~ u=\sqrt{k_{00}-||v||^2}.
\label{eq:lemma1}
\end{equation}
\end{lemma}\label{lemma1}
\begin{proof}
For $B'$ defined by (\ref{eq:lemma1}), we have that
\begin{equation}
B'B'^T = 
\begin{bmatrix} u^2+\|v\|^2 & v^TB^T\\ 
Bv& BB^T\end{bmatrix} 
=\begin{bmatrix} k_{00} & v^TB^T\\ 
Bv& K\end{bmatrix} .
\end{equation}
Since $K'$ is positive semidefinite, $k_0$ must lie in the column space $\mathcal{B}$ of $B$.
%\footnote{
Indeed, if we suppose $k_0$ does not belong to $\mathcal{B}$, then it can be decomposed uniquely as $k_0=s+t$, $s\in\mathcal{B}$ and  $t\in\mathcal{B}^\perp$, with $t\ne 0$. In one hand, $K'$ being semidefinite positive implies that $[1, -at^T]K'[1; -at]=k_{00}-2a||t||^2\ge 0$ for all real value $a$. In the other hand, for $a$ large enough, $k_{00}-a||t||^2\le 0$, which is a contradiction.
%} 
Hence $v=B^\dagger k_0$ is an exact solution of $Bv=k_0$. The fact that $k_{00}-\|v\|^2$ is non-negative comes from the fact that the Schur complement $K-k_0 k_0^T / k_{00}$ of $k_{00}$ in $K'$ is itself positive semidefinite.
%\footnote{
Indeed, since the matrix $k_{00}K-k_0k_0^T=B(k_{00}\textbf{Id}_r-vv^T)B^T$ is positive semidefinite and $B$ has rank $r$, $k_{00}\textbf{Id}_r-vv^T$ is also positive semidefinite. Thus $v^T(k_{00}\textbf{Id}_r-vv^T)v = ||v||^2(k_{00}-||v||^2)\ge 0$.
%}
\end{proof}
This lemma allows us to kernelize Equation (\ref{omega:first}) for any positive exemplar $x_0$ and a fixed set of negatives $X$:
%$\beta^\star(x_0,X), \nu^\star(x_0,X) = \underset{\beta\in\RR^{r+1}, \nu\in\RR}{argmin}J'(\beta, \nu)$, for
$\big(\beta^\star(x_0,X),\nu^\star(x_0,X)\big)\in\RR^{r+1}\times\RR$ is the minimizer of $J'(\beta, \nu)$, where
\begin{align}
J'(\beta, \nu)=
 \theta\ l(1, b_0'^T\beta+\nu) +\dfrac{1}{n}\sumi l(-1,b_i'^T\beta+\nu)
+\dfrac{\lambda}{2}\|\beta\|^2,\label{beta:final}
\end{align}
and $b_i'^T$ is the $(i+1)$-th row of $B'$, $i$ in $\{0,1,...,n\}$. In particular, $b_0'=[u; \ v]$ and, for $i>0$, $b_i'=[0; \ b_i]$. Solution $(\beta^\star,\nu^\star)$ can be computed just as before by Equation (\ref{omega:solution}), replacing $x_0$ by $b_0'$, $X$ by the $(r+1)\times n$ matrix $Q$ of columns $b_1', b_2',...,b_n'$ and $\mu$ by $\mu' = \frac{1}{n}\sum_{i=1}^n b_i'$.


\subsection{Matching score}
Once the optimal parameters $(\beta, \nu)$ from (\ref{beta:final}) and the coordinates $u$, $v$ of $b_0'$ from (\ref{eq:lemma1}) have been found\footnote{We drop the ``$\star$'' in this subsection to avoid clutter up the notation.}, they can be used directly for measuring similarity between images. 
Indeed, the corresponding vector $\alpha$ (or, more correctly, \textit{a} corresponding vector of dimension $n+1$) can be computed by $\alpha = P'\beta$, where $P'=B'(B'^TB')^{-1}$  is the pseudoinverse of $B'^T$. This identity can be written as 
\begin{equation}
\begin{bmatrix} \alpha_0 \\ \hat{\alpha} \end{bmatrix} = \begin{bmatrix} \frac{1}{u} & 0^T \\-\frac{1}{u}Pv & P  \end{bmatrix} \begin{bmatrix}\beta_0 \\ \hat{\beta} \end{bmatrix},
\end{equation}
for $P$ pseudoinverse of $B^T$, $\alpha_0$ and $\beta_0$ scalars. The kernalized SLEM of base feature $x_0$ finally reads:
\begin{equation}
h =\alpha_0 \varphi(x_0)+\sum_{i=1}^n \alpha_i \varphi (x_i)+\nu \in H.
\end{equation}
Inspired by the matching score of \cite{ZePe15}, the matching score $s(h,h')$ between $h$ and the encoding $h'=\alpha_0'\varphi(x_0')+\sum_{i=1}^n \alpha_i'\varphi (x_i)+\nu'$ of an other image $x_0'$, is given by (ignoring biases $\nu$ and $\nu'$ which have empirically no influence): 
%\PP{So, biases $\nu$ and $\nu'$ are ignored?}\RAF{Yes, adding it to the classifier does not change the results in a significant way.} 
\begin{equation}
\begin{split}
s(h,h') & = \langle h, h'\rangle \\
		& = \hat{\alpha}^{T} K\hat{\alpha}'+\alpha_0k(X, x_0)^T\hat{\alpha}'+\alpha_0'k(X, x_0')^T\hat{\alpha}+\alpha_0\alpha_0'k(x_0,x_0')\\
		& = \hat{\beta}^T\hat{\beta}'+\frac{\beta_0\beta_0'}{uu'}(k(x_0,x_0')-v^Tv').
\end{split}
\end{equation}



%%% Local Variables:
%%% TeX-master: "main_eccv"
%%% End:
